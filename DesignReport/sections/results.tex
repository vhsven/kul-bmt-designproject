\section{Results and discussion}

\begin{enumerate}
\item welke resultaten (probabiliteitsimages van validatie): algemeen
\item crossvalidatie
\item waar zijn er volgens ons nog verbeteringen aan te brengen?
\end{enumerate}



mission definition:
The comparative study of \cite{ginneken} indicated the importance of the
algorithm - dataset match. From a number of studies the sensitivities at seven
levels of false positive detection were calculated and then averaged. The best
performing method in this study yielded an average sensitivity of 0.632 for the
detection of all kinds of nodules. The sensitivity per nodule type was also
provided: small nodules (0.634), large nodules (0.628), isolated nodules
(0.609), vascular nodules (0.693), pleural nodules (0.435) and peri-fissural
nodules (0.766).
These results also show that the ease of nodule detection also depends on the
type of nodule. As this information is not available in the annotations of the
scans and as we did not cooperate with a radiologist, it is not possible to
differentiate between the nodules in this project.

Therefore, our aim will be to
detect all types of nodules, regardless of their size or anatomical location.
However, the annotations assign a probability of malignancy for each nodule.
Separating the detected nodules into a malignancy or benignancy class is not the
main aim of this project, but this might be implemented as an extra feature.
In the ideal case, the algorithm would be able to do the processing in a couple
of minutes. This would be very interesting for a commercial software product.
However, considering the available computational power (a laptop) and the
scripting language that is used, this would not be feasible. Python is an
interpreted language which makes it inherently slower than compiled languages
such as C++. Nevertheless, Python was chosen for its rapid prototyping
abilities. Future work may implement our algorithm in C++ or another compiled
language to speed up the computational process.
