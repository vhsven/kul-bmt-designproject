\section{Mechanical Analysis}
\label{sec:analysis}
In the following paragraphs the effects of the impact of the car on the femur of
the pedestrian are investigated. First an estimation of the forces exerted on
the femur is made. Then a force diagram and torque diagrams are calculated.
Based on these diagrams the stresses in the bone are estimated and by comparing
the stresses with experimental data of previous research the probability that
the bone will fail is estimated.

\subsection{Calculating impact force}
We start from the conservation of momentum law in the direction the car is
traveling. This assumes a perfect ellastic collision, which is a very
crude approximation of reality. Because the velocity of the pedestrian in this
direction is zero, so is his momentum. The video clips show that the car and the
pedestrian cling together during the first few seconds after time of impact, so
we consider them as one system with one momentum. %TODO video clips
\begin{equation}
	p_{C_0} + 0 = p_1
\end{equation}

% \begin{equation}
% 	\frac{m_C \cdot v_{C_0}^2}{2} + \frac{m_P \cdot v_{P_0}^2}{2}  = 
% 	\frac{m_C \cdot v_{C_1}^2}{2} + \frac{m_P \cdot v_{P_1}^2}{2}
% \end{equation}

\begin{equation}
	m_C \cdot v_{C_0} + 0 = (m_C + m_P) \cdot v_1
\end{equation}

Solve for $v_1$.
\begin{equation}
	v_1 = \frac{m_C}{m_C + m_P} \cdot v_{C_0} 
	= \frac{2400\text{kg}}{2480\text{kg}} \cdot 10 \tfrac{\text{m}}{\text{s}}
	\approx 9.68 \tfrac{\text{m}}{\text{s}}
\end{equation}

From this, we can calculate the force acting on the pedestrian.
\begin{equation}
	F = \frac{\Delta p}{\Delta t}
\end{equation}

The equation shows that the force $F_P$ is also dependent on the time interval
$\Delta t$ in which the collision occurs. When two hard materials hit each
other, e.g. a club hitting a golfbal, this interval is approximately $400 \mu
s$\footnote{http://www.golfswing.com.au/139}. On the other hand, the typical
impact time - from full speed to standstill - of a car crashing into a wall is
about 100ms \cite{crash_mech}
\footnote{http://auto.howstuffworks.com/car-driving-safety/accidents-hazardous-conditions/crash-test.htm}.
This interval is larger because of crumple zones. However, because our car is
equipped with a bullbar, and because we only take into account the initial
event, we estimate $\Delta t \approx 10$ms.

Plugging this value into the equation gives:
\begin{equation}
	F = \frac{80\text{kg} \cdot 9.68\text{m/s}}{0.01\text{s}} = 77440\text{N}
\end{equation}

We assumed a soft tissue damping factor of roughly 15\%, which allows us to
calculate the resulting foce on the femur:
\begin{equation}
	F_f = 0.85F = 65824N
\end{equation}

\subsection{Discussion on force and torque diagrams}

In \autoref{fig:hyper1} the complex system of the leg of the pedestrian getting
hit by the bull bar of the car is visualised. In order to make a mechanical analysis of
this system a number of assumptions had to be made. To make the assumptions
three things were taken into consideration. First, \cite{snedeker2005assessing}
proposed a set of boundary conditions to analyse the impact of a car bumper on the leg of
a pedestrian. In this paper it is suggested to assume the pelvis to be �fixed�
-- such that the pelvis has a horizontal velocity of zero -- during the first
moments of the crash. Second, we analysed the deformation of a set of
mechanical systems and compared these with our visual analysis of the legs of
dummies in crash tests (REF). The main parts in all mechanical systems were the
pelvis (suggestions: clamped, simply supported or hinge), the femur (suggestion:
a beam), the knee (suggestions: hinge, simple supported or simply a
uninterrupted beam which represents femur, knee and bones in lower leg), the
bones in the lower leg (a beam) and the foot and ankle (suggestions: clamped,
simply supported or hinge). For all combination of the mechanical
representations of these main mechanical parts the line of deformation was
drawn. Finally, the system represented in \autoref{fig:hyper1} was chosen.

%TODO check scharnieren

 \begin{figure}[htp]
\begin{center}
  \includegraphics[page=1,width=\textwidth]{img/hyper.pdf}
  \caption{Schematic of the system under consideration. The pelvis (left) is
  modeled as a clamp, while the angle (right) is replaced by a hinge. The knee
  (circle) is merely added for visual assistance, it does not play any
  significant part in this analysis. The total length $L$ is split up in $L_1$
  and $L_2$ at the point of impact.}
  \label{fig:hyper1}
\end{center}
\end{figure}

After the mechanical system was decided upon, the force and moment equilibrium
equations were calculated. 

\begin{equation}
	\Sigma F=0: R_{Ay} + R_{By} - F = 0
\end{equation}

\begin{equation}
	\Sigma M_B=0: L_2 F - (L_1 + L_2)R_{Ay} - M_A = 0 
\end{equation}

From these equations it is clear we have a hyperstatic sytem of level 1, which
means we have three unknowns and only two equations. Therefore, this system has
to be solved using �the method of the chord�. The first step in applying this
technique is choosing two points $a$ and $b$ that are fixed in your system. Then
a point $c$ is determined which is situated between point $a$ and $b$ and for
which you want to calculate the deformation in the system due to external forces
and moments. In this case point $c$ coincides with point $a$. Next the torque
diagram $M$, taking into account all external forces, is calculated. From this
moment diagram $M$ the reduced moment $M_{red}$ is deduced by dividing M by the
multiplication of the Young�s modulus and the moment of inertia of the system
($EI$). In this way the stiffness of the system is also taken into account which
will allow us to get an exact solution for the hyperstatic mechanical system.
After this the reduced moment diagram is treated as a distributed force acting
on the entire length of the beams. To calculate the equivalent forces of these
distributed forces the reduced moment diagram is divided into triangular parts
of which an $F_{eq}$ is calculated each time. Then the reaction force in point
$c$ is determined. The method of the chord assumes the reaction force in this
point to be the same as the angle of deformation in that point.

In order to solve the hyperstatic system it has to be divided into two
subsystems: a main system (\autoref{fig:hyper2main}) and a recovery system
(\autoref{fig:hyper3recovery}). The main system is the hyperstatic system made
static again by cancelling one force or moment. In this case we chose to cancel
the moment of the clamped pelvis. The task of the recovery system is to restore
the distorted line of deformation of the main system to the line of deformation
of the initial hyperstatic system. Therefore, a moment $m_a$ is added in the
recovery system. The method of the chord is applied on both systems and then the
results of the two systems are superimposed to obtain the results for the
initial hyperstatic system.

 \begin{figure}[htp]
\begin{center}
  \includegraphics[page=2,width=\textwidth]{img/hyper.pdf}
  \caption{The main system (HS).}
  \label{fig:hyper2main}
\end{center}
\end{figure}

\begin{figure}[htp]
\begin{center}
  \includegraphics[page=3,width=\textwidth]{img/hyper.pdf}
  \caption{The recovery system (hs).}
  \label{fig:hyper3recovery}
\end{center}
\end{figure}

\begin{figure}[htp]
\begin{center}
  \includegraphics[page=4,width=\textwidth]{img/hyper.pdf}
  \caption{Reduced torque diagram $M_{red}$ of the main system.}
  \label{fig:hyper4}
\end{center}
\end{figure}

First, we calculate $R_A$ (and thus $\theta_{HS}$) in \autoref{fig:hyper4}.
\begin{equation}
	M_b = M(L_1) = R_A \cdot L_1 = \frac{-L_1 \cdot L_2}{L_1 + L_2} \cdot F
\end{equation}

\begin{equation}
	M_{b,red} = \frac{-L_1 \cdot L_2}{L_1 + L_2} \cdot \frac{F}{EI}
\end{equation}

\begin{equation}
	\Sigma M=0: \frac{2}{3} L_2 F_{e_2} + (L_2 + \frac{1}{3}L_1) F_{e_1} - R_A (L_1
	+ L_2) = 0
\end{equation}

Solve for $R_A$:
\begin{equation}\label{eq:RA}
	R_A = \frac{\frac{2}{3} L_2 F_{e_2} + L_2 F_{e_1} + \frac{1}{3}L_1 F_{e_1}}{L_1 +
	L_2} = \theta_{HS}
\end{equation}

We also know $F_{e_1}$ and $F_{e_2}$: 
\begin{equation}
	F_{e_1} = \frac{-L_1 L_2}{L_1+ L_2} \frac{F}{EI} \frac{L_1}{2}
\end{equation}

\begin{equation}
	F_{e_2} = \frac{-L_1 L_2}{L_1+ L_2} \frac{F}{EI} \frac{L_2}{2}
\end{equation}

Plugging these equations in \autoref{eq:RA} gives:
\begin{equation}
	R_A = \frac{-1}{3} \frac{L_1 L_2^3 F}{(L_1 + L_2)^2 EI} - \frac{L_1^2 L_2^2
	F}{(L_1 + L_2)^2 2EI} - \frac{-1}{6} \frac{L_1^3 L_2 F}{(L_1 + L_2)^2 EI} =
	\theta_{HS}
\end{equation}

\begin{figure}[htp]
\begin{center}
  \includegraphics[page=5,width=\textwidth]{img/hyper.pdf}
  \caption{Reduced torque diagram of the recovery system.}
  \label{fig:hyper5}
\end{center}
\end{figure}

Next, we calculate $R_a$ (and thus $\theta_{hs}$) in \autoref{fig:hyper5}.
\begin{equation}
	M_b = -m = -m_a
\end{equation}

\begin{equation}
	M_{b,red} = \frac{-m}{EI}
\end{equation}

\begin{equation}
	\Sigma M_b = 0: \frac{2}{3}(L_1 + L_2)F_e - (L_1+L_2) R_a = 0 
\end{equation}

\begin{equation}
	R_a = \frac{\frac{2}{3}(L_1 + L_2)F_e}{L_1 + L_2} = \frac{2}{3} F_e
\end{equation}

\begin{equation}
	F_e = \frac{-m}{EI} \frac{L_1 + L_2}{2}
\end{equation}

\begin{equation}
	R_a = \frac{-2}{3} \frac{m}{EI} \frac{L_1 + L_2}{2} = \theta_{hs}
\end{equation}

Now, the following equation must hold:

\begin{equation}
	\theta_{HS} + \theta_{hs} = 0
\end{equation}

\begin{equation}
	\frac{-1}{3} \frac{L_1 L_2^3 F}{(L_1 + L_2)^2 EI} - \frac{L_1^2 L_2^2
	F}{(L_1 + L_2)^2 2EI} - \frac{1}{6} \frac{L_1^3 L_2 F}{(L_1 + L_2)^2 EI} -
	\frac{2}{3} \frac{m}{EI} \frac{L_1 + L_2}{2} = 0
\end{equation}

Solve for m:
\begin{equation}
	m = \frac{-2 L_1 L_2^3F - 3 L_1^2 L_2^2 - L_1^3 L_2 F}{2(L_1 + L_2)^3}
\end{equation}

Using $L_1 = 0.154$m, $L_2 = 0.8$m and $F = 65824$N, we can calculate $M_A$:
\begin{equation}
	M_A = -m = 7814.44\text{Nm}
\end{equation}

With this knowledge, we can finally solve the complete system and construct the
accompanying force (\autoref{fig:hyper6}) and torque (\autoref{fig:hyper7})
diagrams.

%TODO so why did we calculate M_A in the first place?
\begin{equation}
	R_{Ay} = \frac{L_2}{L_1 + L_2} F = 55 198.32
\end{equation}

\begin{equation}
	R_{By} = \frac{L_1}{L_1 + L_2} F = 10 625.68
\end{equation}

\begin{equation} %TODO M what?
	 M = \frac{L_1 L_2}{L_1 + L_2} F = 8 500.54
\end{equation}

\begin{equation} %TODO M what?
	 M = \frac{L_2 M_A}{L_1 + L_2} = 6 552.98
\end{equation}

\begin{figure}[htp]
\begin{center}
  \includegraphics[page=6,width=\textwidth]{img/hyper.pdf}
  \caption{Hyper}
  \label{fig:hyper6}
\end{center}
\end{figure}

\begin{figure}[htp]
\begin{center}
  \includegraphics[page=7,width=\textwidth]{img/hyper.pdf}
  \caption{Hyper}
  \label{fig:hyper7}
\end{center}
\end{figure}

The forces applied on the upper part of the upper leg are definitely the
largest: they are five times larger than the forces applied on the lower parts.
Nevertheless, both distributed forces are of significant magnitude: 65 kN  and
12.5 kN. As they are exerted in the transverse direction on the femur, it is
very likely they will cause a fracture of the bone. All long bones have
anisotropic structure characteristics. This causes them to be stronger in the
longitudinal direction than in the transverse direction. The femur is one of the
strongest bones in the body and is able to carry loads up to thirty times the
body weight. (VERSCHILLENDE KEREN GELEZEN MAAR VINDT REF NIET MEER ) In our
case this means the femur can withstand a load of 23 544 N in the longitudinal
direction.

The moments due to the applied forces are also of significant magnitude and will
therefore cause the bone to bend. Of course, very soon the bone will fracture
under these loading conditions.

\subsection{Calculating stresses in the femur}
To calculate the stresses, we make use of Hertz theory. We model both the
femur and the bullbar as cylinders, and assume they are in direct contact. This
theory requires three main assumptions to hold true:
\begin{itemize}
  \item both materials must have a similar Young's modulus
  \item both surfaces must deform
  \item the contact area must be relatively small compared to the two bodies in
  contact
\end{itemize}

The first assumption will not hold in the case of a stainless steel bull bar,
but it can work for an aluminium bull bar. Furthermore, we estimate the
diameter of the femur and bull bar as respectively $D_{bone} = 24.3$mm and
$D_{bar} = 60$mm.

\cite{puttock1969elastic} suggests following formule to calculate the total
elastic compression at the contact surface, measured along the line of the
applied force $F$.
\begin{equation}
	d = \frac{3\pi^\frac{2}{3}}{2} \cdot F^\frac{2}{3} \cdot (\nu_1 +
	\nu_2)^\frac{2}{3} \cdot \frac{1}{D}^\frac{1}{3}
\end{equation}

\begin{equation}
	v = \frac{1 - \nu^2}{\pi E}
\end{equation}

%TODO G=?
Poisson ratio:
\begin{equation}
	\nu = \frac{E}{2G - 1}
\end{equation}

Contact area radius:
\begin{equation}
	a = \sqrt{Rd} \text{ with } \frac{1}{R} = \frac{1}{R_1} + \frac{1}{R_2} 
\end{equation}

Poisson ratio SS = 0,305
\footnote{\url{http://www.engineeringtoolbox.com/poissons-ratio-d_1224.html}}

E en poisson ratio femur: \cite{huiskes1977geometrical}


\subsection{Will the femur break?}
Compare to the table in \autoref{fig:femurprop}: transverse compression
of max 133 MPa for femur.

Calculated transverse stress: 3.3 GPa
Femural bone will fail!

\begin{figure}[htp]
\begin{center}
  \includegraphics{img/properties_femur.png}
  \caption{Properties of the femur. \cite{Ob}}
  \label{fig:femurprop}
\end{center}
\end{figure}