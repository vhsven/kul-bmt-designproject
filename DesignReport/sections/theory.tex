% voor Kim: http://thesaurus.com/browse/therefore

\section{Theoretical background}
\subsection{Features}
A 'feature' represents a characteristic of a voxel and its environment. Lists of
features for all the voxels in an image are used to describe the entire image.

\subsubsection{Feature extraction}
Feature extraction involves obtaining a list of feature from a resource -- an
image or another dataset -- to describe that dataset in an accurate way that
allows it to be analysed by proper software.

In image processing a wide range of features can be extracted. Some features are
very trivial: the grey value or colour of pixels, the position of pixels, etc.
Features can be shape based: template matching, blob detection, edge detection,
detection of lines and circles by means of a Hough transform, etc.
Once certain objects are detected in the image, their area, curvature and size
can be determined. On the other hand, the shape based features need an a priori
nodule segmentation. Performing such a segmentation is not trivial and
error-prone. Therefore, the features in this project will be selected in a way
there is no need for performing a nodule selection in advance.

We will now highlight some of the most relevant features for our purpose.

\paragraph{Distance Map}
A binary image containing either foreground (1) or background (0) voxels can be
transformed into a distance map. In this process, every foreground voxel is
assigned the distance to the nearest background voxel, while the background
voxels are simply set to zero. In that regard, distance maps can be seen as
generalisations of edges: they contain the same information, and more. An
example of a distance map can be found in \autoref{fig:distmap}.

\begin{figure}[htp] \begin{center}  
\includegraphics[width=\linewidth]{img/distmap.png}   \caption{Example of a
distance map}   \label{fig:distmap} \end{center} \end{figure}

The distance metric must not necessarily be the Euclidean distance. Other
metrics such as the Manhattan distance are also acceptable and easier to compute.

\paragraph{Laplacian}\label{sec:laplace_theory}
One common feature in nodule detection is the laplacian -- also called blob
detector. The laplacian operator applied to a continuous 3D function is defined
as:

\begin{equation}
	\nabla^2f(x,y,z) = \left(\frac{\partial^2 f}{\partial x^2} + \frac{\partial^2
	f}{\partial y^2} + \frac{\partial^2 f}{\partial z^2}\right)
\end{equation}

To be applied to images, it must first be discretized into a 3D convolution
mask. That mask typically has a large negative number in the center, surrounded
by positive ones. As we are dealing with second derivatives, this operation
is very sensitive to noise. One solution to this problem is to convolve the
image with a gaussian kernel of scale $t$ first. Because this kernel has a
low-pass effect, noise will be reduced.

\begin{equation}
	G_t(x,y,z) = \frac{1}{(\sqrt{2\pi} t)^3}e^{-\frac{x^2+y^2+z^2}{2t^2}}
\end{equation}

It can be represented by bionomial filters -- repeated convolutions of [1 1]
with itself -- in the discrete domain, making it rather cheap operation
computation-wise. Convolution also has some interesting properties, which allow
this calculation to be further optimized:

\begin{equation}
	\nabla^2[G_t(x,y,z) * f(x,y,z)] = \nabla^2 G_t(x,y,z) * f(x,y,z)
\end{equation}

The $\nabla^2 G_t(x,y,z)$ term is often referred to as the Laplacian-of-Gaussian
(LoG) or Mexican hat filter. The LoG can be used to detect edges by finding its
zero crossings, but we are more interested in its blob detection capabilities.
It results in a strong positive response for dark blobs of uniform intensity
with extent $\sqrt{2t}$ on a light background and vice versa for light blobs on
dark background. The latter description fits most, if not all, nodules.

To get this strong response, we must carefully calibrate the scale parameter $t$
to the nodule radius. However, because nodules come in different sizes, $t$ will
likewise have to range over different values. In general, larger $t$ values
yield lower filter responses. This makes finding minima and maxima in
scale-space difficult. Using the scale-normalised laplacian operator solves this
problem by multiplying the response with $t$. To make this computationally
expensive operation more bearable, the laplacian is often approximated by the
difference between two levels in a gaussian scale pyramid. The result is called
the Difference-of-Gaussians (DoG).

Then there is one last technicality we must address. The above formulas assumed
that the blobs were isotropic and thus that the LoG could be as well. However,
because we perform most of our calculations in voxel space and voxels are not
perfect cubes, this assumption does not hold. To compensate for this fact, we
define a unique sigma per dimension that is a rescaled variant of the original
based on the corresponding voxel dimension.

Alternatively to the laplacian, one can also use the Hessian matrix of second
partial derivatives as a feature. The laplacian is simply the sum of the
elements on the Hessian's main diagonal. In that sense the latter is more
complete, but also much more computationally expensive. For this reason, we
stick with the laplacian operator.

\paragraph{3D Averaging}\label{sec:3davg} 
This method \cite{keshani} is based on the fact that nodules are more or less
spherical while bronchi and bronchioles are oblong. This might not be visible in
a 2D image where both appear as spheres depending on the orientation of the
bronchioles, but it certainly is if a 3D volume around a finding is taken into
account. Nodules will repeat themselves in the preceeding and/or succeeding
slices at the same place while the smaller bronchi and bronchioles will not. The
comparison between slices was made by calculating the mean value of a window in
the target slice and the average of the mean values of the same window in the
succeeding and preceeding slices:
\begin{equation}
M_{ij}^p = \frac{1}{9} \sum_{k,l = -1}^{+1} L^p(i+k,j+l)
\end{equation}
\begin{equation}
M_{ij}^{+} = \frac{1}{q} \sum_{p=p+1}^{p+q} M^p_{ij}
\end{equation}
\begin{equation}
M_{ij}^{-} = \frac{1}{q} \sum_{p=p-q}^{p-1} M^p_{ij}
\end{equation}
\begin{equation}
\text{3D averaging} = M^{-}_{ij} M^{+}_{ij} 
\end{equation}
The index $p$ indicates the z-index of the slice and the parameter $q$ is the number
of slices that are considered. The latter parameter is calculated as the ratio
of the estimated nodule length to the thickness of the slices.
\begin{equation}
q=\frac{c}{T}
\end{equation}
To differentiate between nodules and small bronchioles the parameter $c$ was set
at 5mm and 2,5mm. The window around each voxel was set at $3 \times 3$. In case
of a high $q$ a high 3D averaging score will classify a voxel as belonging to a
nodule whereas a low score will classify a voxel as a non-nodule.

\subsubsection{Feature selection}
A problem that may arise when analysing a dataset based on list of features, is
that the amount of features is too large. Performing an analysis on large
amounts of datasets requires a large amount of memory and computational power.
Furthermore, the classification algorithm may overfit the training data and will
not be able to generalise anymore. Therefore, a range of dimension reduction
techniques can be applied to extract the uncorrelated and most important
features of a list. A second option is determining the best features in an
empirical way. A probability image can be used for a visual inspection.

\subsubsection{Features in nodule detection} \label{sec:featureselection}
A non-exhaustive literature review revealed some commonly used features used in
automatic nodule detection. \cite{tartar} primarily used morphological features:
area, perimeter, diameter, solidity, eccentricity, aspect ratio, compactness,
roundness, circularity and ellipticity. To select these features the minimum
Redundancy Maximum Relevance (mRMR) method was applied. Selecting relevant
features is import to improve the accuracy of the algorithm and to reduce the
processing time. \cite{mur} used 3D local image features which were calculated
per voxel: shape index and curvedness. \cite{chen} calculated the size, margins,
contours and internal characteristics of the candidate nodules. \cite{keshani}
used 2D stochastic features �- grey level values and intensity values --  as
well as 3D anatomical features to remove the bronchioles from the list of candidate
nodules. The features selected by \cite{teramoto} were area, surface area,
volume, CT value, convergence, diameter and overlapping area. The algorithm of
\cite{ozekes} implemented 3D features such as straightness, thickness, vertical
and horizontal widths, regularity and vertical and horizontal black pixel
ratios.

These features are calculated based on a prior segmentation of the nodules. On
the other hand, instead of doing a nodule segmentation first, a lot of features
can be calculated on the image itself based on grey values, intensities,
grey values in the neighbourhood, etc. Although this is not a very common
approach, these features can be generated without any preprocessing of the
image.

\subsection{Machine learning}
The research field of machine learning is dedicated to the automatic
learning of software in order to make accurate predictions based on past
observations \cite{mach}. This concept is of course very interesting when
detecting nodules as a vast amount of 'past observations' is available.
Classifiers are algorithms that classify given examples into a given set of
categories \cite{mach}. CAD systems which implement a classifier tend to
outperform the CAD systems which do not \cite{lee2010}. Therefore, the use of
classifiers is very common in this field of research and many classifiers have
been tested.

\subsubsection{Classifiers used in nodule detection systems}
\cite{caruana} compared the performance of a range of classifiers on eleven
binary classification problems. The supervised learning algorithms that were
used are SVM, NN, Naive Bayes, Memory-Based Learning, RF, Decision Trees, Bagged
Trees, Boosted Trees and Boosted Stumps. Prior to calibration Bagged Trees, RF
and NN perform the best on average across all test problems. After calibration
Boosted Trees outperformed all other methods. The performances of SVM, Boosted
Stumps and Naive Bayes were also dramatically improved by calibration. The
performance of RF was not increased significantly. Overall, \cite{caruana}
suggests calibrated boosted trees is the best learning algorithm. RF are close
second, followed by uncalibrated bagged trees, calibrated SVM and uncalibrated
NN. However, the training of Boosted Trees is inherently sequential which makes
it slower to implement than RF. Another problem may be the noisyness of the
data. Therefore, the choice and tuning of the parameters of the boosted trees
(depth of trees, amount of trees, etc.) algorithm should be done carefully and
this will take some time.

SVM are widely used in the development of nodule detection CAD systems
\cite{keshani, lee2010, ozekes}. According to \cite{ash} SVM even outperforms
RF. However, SVM has some disadvantages. First of all, the featues
need to be determined in advance and there is no such thing as a standard
feature set (see also section \ref{sec:featureselection}). In the ideal case these
features should all have the same dimensions and/or magnitudes. With SVM
problems may arise with noisy data and images are very often quite noisy. The
time complexity of SVM is $O(n^2)$ \cite{svmcompex}.

The best performing CAD system in the ANODE09 challenge was the ISI-CAD
algorithm \cite{ginneken}. ISI-CAD used a k-nearest neighbour classifier to
reduce the amount of FP, but this method also has the disadvantage that the
features should all be of the same magnitude in order to perform an optimal classification.

 
\subsubsection{Random Forests, an ensemble classifier }
Ensemble learners combine decisions of multiple classifiers to form an
integrated output \cite{lee2010}. The use of multiple learning algorithms at the
same time has the advantage that a better predictive performance is obtained
compared to the performance demonstrated by each individual learning
algorithm separately.

Random Forests (RF) is a relatively new classification method which has not been
exhaustively explored yet. ``Random Forests are a combination of tree predictors
such that each tree depends on the values of a random vector sampled
independently and with the same distribution for all trees in the forest. The
generalisation error for forests converges to a limit as the number of trees in
the forest becomes large.'' \cite[~p.5]{breiman} For this reason RF is an ensemble
learning method. So given a test sample as the input, this input vector is put
down each of the trees, each tree gives a classification and the forest selects
the classification that has most votes. Consequently, the output of the RF
depends on the combination of results from all individual trees. In this way a
variance reduction is achieved and the output is made more robust against noise
\cite{breiman, lee2010, RFcompex}. RF has the advantage that the set of features
does not need to be known in advance as the algorithm itself decides on itself
which features to use. Therefore, a lot of features can be generated at will and
the algorithm itself will decide on using them or not. The time complexity of RF
is $O(n \log n)$ \cite{RFcompex} which makes it more suitable than e.g. SVM for
large datasets.

However, although the RF classifier might be a good choice for analysing large
datasets, the amount of memory and computational power can still be a problem.
Hence, an optimisation of the implementation of the RF classifier is
desirable. A possible optimisation is the use of a cascaded classifier.

\subsubsection{Cascaded classifiers}
A cascaded classifier exists of several classifiers at different levels that are
concatenated. Consequently, it is a special case of an ensemble classifier. The
cascaded classifier uses all the information that is obtained in a previous
level to provide the classifier in the next level with additional info on the
data. On a lower level in a cascaded classifier the amount of features and the
complexity of the features that are used are lower. For example, the classifier
at the first level may determine a threshold to separate nodule voxels from
non-nodule voxels based on their grey values. Using this trivial feature a large
part of the voxels can already be eliminated as their grey value is too high or
too low to be a nodule voxel. The features on the second level are more complex
and therefore require more computational power. However, this is not a problem
anymore as the amount of voxels to be processed was reduced after the first
level. At the second level again a number of voxels are eliminated. The number
of levels, and therefore the number of features, can be increased until the end
results are satisfying. The reason for using a cascaded classifier is that both
memory and CPU time are often limited. By discarding a certain amount of
non-nodules at each level of the classifier, it is not necessary to calculate
all features for all nodules which saves memory and CPU time.

The performance of the classifier depends on the features that are implemented
on each level. For data mining the general rule is the more the better. If a lot
a features are provided for each voxel the classifier can set more thresholds
and is therefore better able to separate different classes. We started off from
this concept, but very soon we had to deal with memory errors in Python. The
code was therefore optimised, but still we had to cut in the amount of features
that were used in the final version of the classifier as the amount of memory
available remained a problem during the whole project.

\subsection{Validation metrics}
In order to evaluate the performance of a binary classifier, we introduce some
statistical concepts. The reader should be familiar with Type I and Type II
errors. A Type I error occurs when the model predicts something to be there
while in reality it is not. In this text we call these occurrences false
positives (FP). In our scenario, this corresponds with a classifier indicating
that a nodule is present when there is really none.

Vice versa, a Type II error occurs when the model predicts something to be absent
when in reality it present. We call them false negatives (FN). False negatives
in our scenario represent nodules not detected by the classifier.

On the other hand, true positives (TP) and true negatives (TN) represent the
cases where the classifier properly detected the presence or absence of the
nodule respectively.

\begin{table}[ht]
\begin{center}
	\begin{tabular}{r | c c}
						& Nodule 	& Non-Nodule \\
		    \hline
		    Positive 	& TP 		& FP\\
		    Negative 	& FN 		& TN \\
	\end{tabular}
	\caption{Summary of some basic statistical measures.}
	\label{tbl:stats}
\end{center}
\end{table}

\autoref{tbl:stats} summarizes these definitions, it is sometimes called the
confusion matrix.

Because the terms above are in absolute numbers, they are difficult to compare
across studies. That is where sensitivity and precision come in. Sensitivity
compares the amount of true positives with the total amount of actual positives.
Synonyms include the true positive rate or the recall rate. Precision also
focuses on the true positives, but this time in comparison to the total
predicted positives.

\begin{equation}
	\text{sensitivity} = \frac{TP}{TP + FN}
\end{equation}

\begin{equation}
	\text{precision} = \frac{TP}{TP + FP}
\end{equation}

Ideally, both measures should be 100\% but that is an unrealistic expectation.
There is an inherent trade-off between the two. For example when the
algorithm is made more strict in order to detect less false positives, precision
rises. At the same time the amount of false negatives likely increase as well,
causing sensitivity to drop. In our case there is a clear preference for a higher
sensitivity, even though it may cost us some precision.

One other important measure is the accuracy. It is the ratio of all correctly
classified occurrences over all occurrences. Note that it is much less
meaningful in unbalanced classifications like ours where one class is far more
prevalent than the other. Indeed, even a trivial classifier always returning the
most common class can score very high on this. Alternatives for accuracy include
the $F_1$ score and the log loss, but these are considered out of scope for the
project.

\begin{equation}
	\text{accuracy} = \frac{TP + TN}{TP + FP + TN + FN}
\end{equation}

Ideally TP, FP, TN and FN should all be measured in the same units: either
voxels or nodules. Pure voxel classification schemes can simply calculate how
many voxels were correctly identified, while classification schemes working with
nodules can use the number of nodules. However, our approach is a hybrid of the
two. We do not require all voxels of a nodule to label the nodule as TP, just a
couple should be enough. Hence, our TP and FN are expressed in number of
nodules. This leaves us with a problem for FP and TN, as these cannot be
expressed in an amount of nodules. We somewhat alleviate this problem by
grouping clusters of voxels together, turning them into ``potential nodules''.