\section{Literature review}
\subsection{Introduction}
In order to establish a well thought pulmonary lung CAD system, the appearance
of lung abnormalities is discussed and a delineation of the term 'nodule' is
provided. Then an overview of commercial and research-based CAD systems is given
and a comparison of performances is discussed.

\subsection{The biology of lung nodules}
Pulmonary odules are lung tissue abnormalities that are roughly spherical with a
diameter up to 30 mm. On chest CT scans they appear as a rounded or irregular
opacity (\autoref{fig:nodules}). Many types of lung nodules can by distinguished
on CT scans. A centrilobular nodule is separated by several millimeters from the
pleural surfaces, fissures and interlobular septa. Micronodules are very small
nodules and have a diameter less than 3 millimeters. A ground-glass nodule -- or
non-solid nodule -- appears on the CT scans as a hazy attenuation in the lung.
This type of nodule does not efface the bronchial and vascular margins. A solid
nodule shows a homogeneous soft-tissue attenuation. Finally, a part-solid nodule
exhibits both ground-glass and solid soft-tissue attenuation characteristics
\cite{nodule}.
\begin{figure}[htp]
\begin{center}
  \includegraphics[width=\linewidth]{img/vanginnekennodules.png}
  \caption{In every box a nodule is shown in a sagittal, coronal and axial
  view. The diameter is provided between brackets. (a) isolated nodule (4,4
  mm); (b) pleural nodule (4,2 mm); (c) peri-fissural nodule (4,8 mm); (d) nodule with vascular
  attachments (5,9 mm); (e) ground-glass nodule (5,4 mm); (f) large pleural
  nodule (18,4 mm); (g) small nodule (3,2 mm); (h) small nodule (3,5 mm); (i)
  small nodule (2,3 mm); (j�l) shows three examples of bright, calcified nodules
  (indication of benign nodules)}
  \label{fig:nodules}
\end{center}
\end{figure}
The types of nodules stated above can be categorised. Juxta-vascular
pulmonary nodules have significant connections to their neighbouring vessels.
Pleural tail nodules have only thin connections to the neighbouring pleural
wall. Well-circumscribed nodules on the other hand do not have a connection to
the neighbouring vessels and structures. Juxta-pleural nodules show some degree
of attachment to their neighbouring pleural surface \cite{kostis}.
\begin{figure}[htp]
\begin{center}
  \includegraphics[width=\linewidth]{img/vanginnekenNonnodules.png}
  \caption{In the three boxes an apical scarring, a pleural thickening and a
nodular abnormality next to an emphysematous bulla are displayed in a sagittal,
coronal and axial view. These may be perceived by a nodule detection algorithm
as nodules, but in fact they are no nodules \cite{ginneken}.}
  \label{fig:Nonnodules}
\end{center}
\end{figure}
A number of nodule segmentation algorithms perform well in detecting specific
types of nodules. However, these CAD systems show large limitations in detecting 
for example non-isolated nodules that are connected to the pulmonary wall \cite{keshani}. 
These algorithms can be useful in particular situations, but if a detection 
algorithms really aims at being an asset for the radiologist, it should be able 
to detect all nodules while rejecting as much false positives (\autoref{fig:Nonnodules}) 
as possible.

\subsection{Overview of existing lung nodule detection systems}
As the demand for a reliable CAD system to detect pulmonary nodules is urgent, a
lot of research has been dedicated to the matter. Several commercial systems
have already been developed and many workstations that radiologists use to
examine CT scans offer on-board nodule detection or enhancement capabilities
\cite{ginneken}. However, although a lot of efforts were made, the results shown
in the various studies are rather diverse.

\subsubsection{Commercial systems}
In 2004 iCAD, Inc., provided lung cancer detection, analysis and tracking
software for the TeraRecon's Aquarius product line. The latter licensed several
software modules from iCAD. The iCAD QuickCue\texttrademark{} automatically
detects cancerous lung nodules. The iCAD QuickMatch\texttrademark{} locates,
compares and tracks nodules in previous or subsequent patient studies
\cite{tera}. However, the ImageChecker CT{}\texttrademark{}, launched by R2
Technology Inc., was the first CAD system approved by the US Food and Drug
Administration (FDA) for the detection of lung nodules during the examination of CT scans
\cite{Mevis}. In 2005 R2 Technology introduced the second-generation
ImageChecker CT Lung Version 2.0 CAD system which also implemented the AutoPoint
temporal comparison algorithm. This CAD system ``highlights abnormalities'' and
compares new and past images to demonstrate changes that have occured over time
\cite{diag, r2}. In 2006 Vital Images, Inc., and R2 Technology announced
the implementation of the R2 Technology's ImageChecker CT Lung CAD software into
the Vitrea workstations \cite{vital}. In the same year another company, Hologic,
Inc., acquired R2 Technology and implemented their CAD technology
\cite{Hologic}. Then, in 2008, MeVis Medical Solutions AG, Inc., acquired the
Pulmonary Computed Tomography Business from Hologic R2, Inc. \cite{Mevis}.

Although the technology from R2 Technology is in high demand, some companies
have developed their own software. In 2007 Medicsight plc, Inc., announced it
was granted a medical device license from the Therapeutic Products Directorate
of Health in Canada to introduce Medicsight LungCAD API\texttrademark \cite{HI}.
Another company, Median Technologies Inc., offers the LMS-Lung and LMS-Lung /
CAD modules which provide quantification and detection functionalities for
pulmonary (solid) nodules and micronodules \cite{median}. Siemens on the other
hand has developed the syngo.CT Lung CAD\texttrademark{}. They claim it is ``a
fully automated computer assisted second reader tool'' that is designed to
assist radiologists in the detection of solid pulmonary nodules \cite{siemens}.

\subsubsection{Publications of automated lung detection systems}
Apart from these numerous commercial systems many academic research
centers have tried to come up with a successful pulmonary nodule detection
system. \cite{review} suggest that most CAD systems for the automated detection
of lung nodules proceed according to a number of steps of which the first one is
the acquisition of data. The detection of lung nodules is preferably performed
on CT scans as they enable the visualisation of small volume and low-contrast
nodules because of the limited slice thichness. A large number of chest CT scans
are available in public databases such as the LIDC/IDRI database. In the next
step, the data are pre-processed to remove noise and artefacts which improves
the quality of the images, but it is not necessary to do so. In the third step a
segmentation of the lungs is performed. The lung lobe region is identified and
the rest of the image is removed. This reduces the computational cost compared
to the case where the whole image is processed and it increases the reliability
and the accuracy of the algorithm. The next step is the nodule detection. ``Lung
nodule detection refers to the process of determining whether nodule patterns
are present in the image and identifying the location of the nodules''
\cite[p.~154]{review}. Nodule detection methods can be categorized according to
the detection method that is applied. The first group of publications uses a
template matching technique to detect specific geometries. A second group relies
entirely on the output of a lung nodule segmentation method. The third group
applies a classification technique to classify voxels or regions of interest
(ROI). In addition, a clustering method may be implemented to improve the
performance of the system. The systems that include a classification component
in their nodule detection algorithm have demonstrated better performances
\cite{review}. In the final step of the process, the amount of false positives
is reduced to achieve a maximum sensitivity.

The first problem that arises when processing CT scan in search for nodules is
that one has to rely on the annotations made by expert radiologists. Accurate
delineation of these lung abnormalities is crucial for optimal image analysis.
The current approach to obtain delineations of lung nodules in CT scans involves
one or more radiologists manually drawing the boundaries of the nodules. Often
this manual segmentation overestimates the nodule volume to ensure the entire
lesion is enclosed \cite{rex}. Furthermore, this process shows a high inter- and
intrareader variability \cite{cooper}. But the success of the extraction of
image features depends on the accurate delineation of the nodules. Therefore, it
is of utmost importance this delineation is done in an accurate and reproducible
way. \cite{gu} have improved the ``Click and Grow'' algorithm that has been
developed by Definiens AG and Merck and Co., Inc., which semi-automatically
isolates tumors in CT images. The idea is that a radiologist detects the nodule
and clicks on the region of interest in a 2D slice. This click initiates
multiple seed points in a certain area. Then the application builds out the
object three-dimensionally by region growing. An ensemble segmentation is
obtained from the multiple regions that were grown. An evaluation on a set of
129 CT scans using a similarity index (SI) was performed.
The average SI was above 93\% which shows stability of the algorithm. The
average SI for two different readers was 79,5\%.

Apart from improving the manual nodule delineation of the radiologists, an image
processing algorithm may also increase their detection rate by assisting as a
``second reader''. \cite{roos} assessed the diagnostic performance of
radiologists -- with their years of experience ranging between 9 and 24 years --
and their temporal variation using incremental CAD assistance. 20 scans
containing 190 non-calcified nodules with a magnitude of 3 mm and above were
examined by three radiologists. After a free search, the radiologists
independently evaluated a number of CAD detections per scan. The average
sensitivity of their free search was about 53\% (range, 44\% - 59\%) at 1,15
false positives (FP) per scan. This increased up to 69\% (range, 59\% - 82\%)
and a FP rate of 1,45 per scan when using the CAD assistance. The increase in
sensitivity, with only a minimal increase in FP, was significant during a time
period of 100 seconds. Then the increase in sensitivity flattened from 14\% to
only 2\%. This evolution was due to the fact that the CAD nodules were presented
to the radiologists in order of CAD score and was not due to a temporal change
in the readers' performance. It was also noticed in this study that
different readers may experience a variable benefit from the use of CAD as some
readers tend to often reject true positive CAD candidates. This reduces the
potential benefit of CAD assistance. Nevertheless, \cite{roos} states that CAD
has the potential to equalise performance among readers by reducing individual
detection errors.

\cite{elbaz} proposed a three step algorithm to isolate lung nodules from spiral
chest low-dose CT scans. In the first step the lung tissues were isolated by
applying an iterative Markov-Gibbs random field (MGRF) based segmentation
framework. To retain the nodules attached to the pulmonary walls, the
segmentation was refined by the iterative conditional mode relaxation that
maximizes a MGRF energy. Then the lung nodules, arteries, veins, bronchi and
bronchioles were separated from the rest of the tissues in the slice. In the
second step the lung nodules (2-12 mm) were detected by applying 3D and 2D
templates which describe typical geometry and greylevel distributions within
nodules of the same type. Four template shapes were used: solid sphere, hollow
sphere, solid circle and solid semicircle. The radius and the greyscale
intensity of the templates were made adaptable. The detection combined the
normalized cross-correlation template matching and a genetic optimization
algorithm. The third step eliminated the FP using three textural and geometric
features that were calculated for each detected nodule.
To distuingish between FP and true positives (TP) Bayesian supervised classifier
statistical characteristics from a training set (20 FP, 20 lung TP, 20 lung wall
TP nodules) were selected from 50 separate subjects. CT scans from 200 subjects
were tested in this study and a sensitivity of 82,3\% and a FP rate of 9,2\%
were reached. The algorithm, that was implemented in C++ on an Intel Dual Core
processor with 16 GB memory, took about 5 minutes to process 182 CT slices of
size 512 x 512.

Other studies rely on a nodule segmentation method to detect lung abnormalities.
\cite{kuh} applies morphological opening, erosion, thresholding, seed
optimisation and boundary refinement operations to extract large nodules.
\cite{itai} proposes a segmentation of the lung areas using SNAKES method which
is an active contour model. Abnormal shadow areas over the size of 5 mm are
classified by using voxel densities. The algorithm was applied on 9 CT scans and
a TP fraction of 0,8 and a false negative (FN) fraction of 0,2 were obtained.

A third category of nodule detection methods are the classification based
methods. The main difference in output between a classification based detection
method and a segmentation method is that the latter will provide the user with a
delineation of the entire nodule, while the former will give a
nodule-probability per voxel. \cite{ozekes} tested four different learning based
classification methods: a Neural Network (NN) classifier, a Support Vector
Machine (SVM) classifier, a Naive Bayes classifier and a logistic regression
classifier. First a number of ROI were extracted by applying thresholding and an
8-directional search in which candidate lung nodule voxels had to have
neighbour voxels with densities between a minimum and maximum density threshold.
From these ROI a number of features were extracted: straightness, thickness,
vertical and horizontal widths, regularity and vertical and horizontal black
pixel ratios. These features were then fed to the four classifiers. The NN
classifier showed the best results, followed by the SVM classifier.
\cite{keshani} also applied a SVM classifier. The lung area was first segmented
by active contour modelling, which was followed by a set of masking techniques
to transfer nodules from non-isolated into isolated ones. Based on a set of 2D and 3D features the
SVM classifier was able to detect the lung nodules. Then the contours of these
nodules were extracted by active contour modelling. In a last step the lung tissues in
the original image were classified into four classes: lung wall, parenchyma,
bronchioles and nodules. The results from this classification were used to
distinguish solitary nodules from attached ones. When this algorithm was used to
detect nodules in the ANODE09 dataset an average detection rate of 37,8\% was
obtained while the best performing method yielded a detection rate of 63,2\%.
The latter algorithm, ISI-CAD, was developed by \cite{mur} and used the local
image features -- shape index and curvature -- to detect candidate nodules. Two
successive k-nearest-neighbour (K-NN) classifiers were applied to reduce the
number of FP. This yielded a sensitivity of 80\% with an average of 4,2 FP per
scan. \cite{sluimer2003} also used k-NN for developing an algorithm which
automatically distinguishes between normal and abnormal lung tissue.
Before the k-NN was applied, a principal texture analysis was performed in this
study to determine local features.

In addition to applying a classification based detection method, a clustering
method can be implemented as well to improve the performance of the classifier.
\cite{kawata} developed a linear discriminant classification boosted by k-means
clustering to distuingish between malignant and benign nodules based on
topological histogram features. The k-means clustering divided the datasets of
nodules in homogeneous classes to improve the performances of the linear
discriminant classifier. \cite{lee2010} also applied an ensemble classification
aided by clustering (CAC) method. First, nodule and non-nodule parts of a
training set were merged and then clustered into M clusters to take advantage of
the similarity among features of nodule and non-nodule instances. Then each
cluster was again divided into two groups, namely nodules and non-nodules. A set
of multi-class classifiers was then trained. The classifiers used were a Random
Forest (RF) classifier, a SVM classifier and a Decision Tree (DT) classifier.
There were two different clustering methods applied: k-means and expectation
maximisation (EM). The best results were obtained by the RF CAC EM algorithm. It
yielded a classification accuray of 97,72\%, while SVM and DT CAC EMs recorded
96,46\% and 93,98\% respectively. However, the execution time of the SVM CAC EM
algorithm was the lowest (182 seconds). These results were compared to non-CAC
methods. A classification accuracy of 95,64\%, a sensitivity of 95\%,
a specificity of 96,28\% and a FP rate of 3,72\% were obtained by non-CAC RF.
This method recorded the lowest execution time (10 seconds).


\subsubsection{Performance of existing systems} \label{sec:performance}
 The algorithms presented in a wide range of papers report varying degrees of
 success in the automated detection of nodules. However, it is very difficult to
 compare studies against one another in a meaningful way due to differences in
 the size of the datasets, the evaluation methods, the data selection criteria
 and the characteristics of the nodules under examination \cite{lee2010}.
 Especially comparing older and contemporary studies is difficult as older ones
 may have used scans with thicker sections (range 2.5 - 10 mm), on which small
 nodules are rather difficult to detect, than the scans nowadays (2,5 mm)
 \cite{lee2010, ginneken, mur}. Some studies focus on nodules below or above a
 certain size or on special types of nodules (e.g. solid nodules). \cite{mur}
 performed an extended literature review and found that the number of scans used
 for testing varied between 5 and 500 with a median number of 29,5. Many of the
 studies included multiple scans from individual patients, which means that the
 diversity of the available nodules was reduced. Furthermore, the results of
 publications are often presented in diverse ways.
 
 In order to improve the access to data, and thereby the comparability between
 studies, the Lung Image Data Consortium (LIDC) created a publically available
 database which provides researchers with a vast amount of test- and
 trainingsdata. Nevertheless, as one can take different subdatasets from this
 large database, it is still difficult to compare results in an objective and
 meaningful way. Therefore, \cite{ginneken} created ANODE09, a database of 55
 scans and a web-based framework which allows researchers to test their
 algorithms and to compare results against one another.
 




