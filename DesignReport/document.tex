\documentclass[titlepage, 10pt]{article}

\usepackage{graphicx} %[pdftex]
\usepackage[hidelinks]{hyperref}
\usepackage{titlepic}
\usepackage{fullpage}
\usepackage{pdfpages}
\usepackage[lined,ruled]{algorithm2e}

\titlepic{\includegraphics[width=0.30\textwidth]{img/sedes.jpg}}
\title
{
	[H03I7A] Design in Medical Technology\\
	Shape-based feature analysis\\
	for nodule detection in lung images
}
\author{Kim Nuyts \and Sven Van Hove}
\date{\today}

\begin{document}

\pagenumbering{roman} %i, ii, iii
\maketitle

\clearpage
\pagenumbering{Roman} %I, II, III
\tableofcontents
\clearpage

% paragraph makeup, after ToC
\setlength{\parindent}{0pt} %don't indent new paragraph
\setlength{\parskip}{2ex} %add blank line in between

\section*{Summary} %\addcontentsline{toc}{section}{Summary}

The goal of this project is to develop a fully automated algorithm for the
detection of pulmonary nodules in CT scans. A cascaded Random Forests classifier
will be used to assign nodule probabilities to individual voxels. To train the
classifier, 30 datasets from the LIDC/IDRI are used. Voxels in the center of the
nodule are used as positives samples, while negative voxels are randomly sampled
from any image part that is at a safe distance from all nodules. For each of
these voxels, a feature vector is generated and fed into the classifier.
Five-fold cross validation combined with an optimal parameter grid search shows
accuracy increases to 98,2\% in the last cascade level while at the same time
it makes sure the classifier's hyperparameters are properly calibrated to
prevent overfitting.

Once the classifier is trained, we use 8 different datasets from the LIDC/IDRI
database to perform tests and validation. For performance reasons, a lung mask
is calculated of each set before actual testing begins. We propose two lung
segmentation algorithms, one based on thresholds and morphological operations
and another based on region growing. Either of these will provide us with the
initial mask to start the calculations on.

The cascaded classifier comprises of four levels. In the first level, intensity
is used as a feature as it is readily available. This teaches the classifier
that nodules belong to the soft tissue window, allowing it to discard any voxel
outside of that range. The second level feature is a laplacian blob detector
over several scales in addition to a distance map. This allows the classifier to
specifically detect nodule-sized blobs in the 3D image while staying clear of
the lung walls. At the third and fourth level a 3D averaging effect is used to
distinguish between small nodules and other structures such as bronchioli or
blood vessels by taking into account the structures extent in previous and
following slices. Many other features were implemented and tested, but
ultimately did not make the cut.

After the last cascade, roughly 0,5\% of all voxels remain. We found on average
2,17 true positives and 4279,43 false positives per dataset, while maintaining
the number of false negatives at zero. This yields a perfect sensitivity of
100\% but a very low precision of 0,0634\%. This number is so low because our
algorithm is not strict enough yet and we have not implemented a proper false
positives reduction method. On top of that, our number of false positives is
expressed in number of voxelclusters (or ``potential nodules'') while the other
two metrics are expressed in proper nodules. This discrepancy makes comparison
between both metrics very difficult.

%TODO hoe deze resultaten zich verhouden tot de literatuur. Daar weet ik ni zo veel van.. :/

To improve these results, future work will have to continue fine-tuning the
parameters of our algorithms and improve the effectiveness of the features. In
particular the effectiveness of the 3D averaging feature was found lacking.
Raising the cascade threshold after level two also looks promising. Other
options include expanding the number of training sets and implementing new
features. Furthermore, future research should look into more advanced voxel
clustering strategies such that the remaining clusters are properly worthy of
the term ``potential nodule''. We believe that this last step alone would
significantly reduce our false positives and consequently increase the
precision.

%TODO distmap needed? result without?
\clearpage
\pagenumbering{arabic} %1, 2, 3

\section{Introduction}

written at the end of project
IN GENERAL
\begin{enumerate}
\item goal of project
\item importance of goal
\item how will goal be achieved
\item overview of main sections in report
\end{enumerate}

In Belgium one man in three and one woman in four faces cancer before his or
her 75th birthday \cite{kanker}. In 2010, 62.017 new cases of cancer were
diagnosed in Belgium \cite{kankerliga}. The World Health Organisation estimates
the worldwide death toll from lung cancer will be 10.000.000 by 2030, which
makes it one of the deadliest cancers \cite{gu, zheng}.
However, an early detection can increase the survival rate up to 70-80\%
\cite{swensen}. Furthermore, research has shown that the detection of lung
cancer in an early stage broadens the amount of treatment options and increases
the amount of invasive surgery\cite{greenlee}.

Due to recent developments in computed tomography (CT) technology it is now
possible to obtain near isotropic, submillimeter resolution images of the
complete chest in a single breath hold. This high resolution has the advantage
that it enables visualisation of small and low-contrast nodules that could
hardly be screened in conventional programs. The downside is that enormous
amounts of data are generated which increases the work load of radiologists,
especially since low-dose CT scans are more and more implemented in routine
screenings. Still, this is no idle measure. Long nodules are very commonly
detected on CT scans. Research shows that up to 51\% of smokers aged 50 years or
older have pulmonary lung nodules on CT scans \cite{mahon}. Therefore, the
United States Preventive Services Task Force for example stated that it
``recommends annual screening for lung cancer with low-dose computed tomography
(LDCT) in adults aged 55 to 80 years who have a 30 pack-year smoking history and
currently smoke or have quit within the past 15 years'' \cite{ups}. Therefore,
the detection of pulmonary nodules from volumetric computed tomography (CT)
scans is one of the most studied CAD applications \cite{sluimer}.

Currently, expert radiologists perform the investigation of the
CT scans. They use the shape, the texture, the location and the growth rate of
the volume of the nodule as clinical parameter to determine the malignancy of
the nodules and to decide on the diagnosis of lung cancer. A jagged shape
nodule is more likely to be lung cancer than a smooth one. A fatty, bony,
watery nodule or a mixture of these different contents is less likely to
indicate lung cancer than a nodule that is attached to a vessel. A nodule
attached to the lung wall is typically diagnosed as benign if the
volume-doubling periode is longer than 400 days \cite{wu}. Nevertheless, the
examination of these scans is a time-consuming task and is not free from errors.
Although small nodules are in principle detectable in CT scans, a non-negligible
fraction may be overlooked if they are situated in a maze of vessels of similar
size \cite{ozekes}. An example of a nodule hidden in a maze of vessels is shown
in \autoref{nodmaze}.
\begin{figure}[htp]
\begin{center}
  \includegraphics[width= 30 mm]{img/noduleMaze.png}
  \caption{Nodule in maze of vessels of similar size \cite{noduleMaze}.}
  \label{nodmaze}
\end{center}
\end{figure}
Another problem that arises is the intra- and interreader variability
amongst radiologists in pulmonary lung detection \cite{armato} \cite{hens}. Therefore,
there is a need for a computer-aided detection (CAD) system that can assist the
radiologist in the detection of pulmonary nodules.

VERDER
contributions of the paper: explore the use of ensemble classifier RF in nodule detection (not first nodule
segmentation)
fast + accurate
assign nodule-probability to each voxel


mission definition:

The goal of this project is to develop a fully automated algorithm for the
detection of lung nodules, based on assumptions about the shape and appearance of their
features. An algorithm based on machine learning techniques will be used to
tackle this goal. One of the challenges is the selection of the features. The
fact that some nodules are embedded in surrounding tissue (e.g.
lung wall, blood vessel, etc.) presents another important challenge.





\section{Literature Review}
\subsection{Introduction}
In Belgium one man in three and one woman in four faces cancer before his or
her 75th birthday \cite{kanker}. In 2010 62017 new cases of cancer were
diagnosed in Belgium \cite{kankerliga}. Lung cancer is the second most common
cander for men, and the third for women.
On top of that, it is one of the deadliest cancers
\cite{zheng}. However, an early detection can increase the survival rate up to 70-80\%
\cite{swensen}. Furthermore, research has shown that the detection of lung
cancer in an early stage broadens the amount of treatment options and increases
the amount of invasive surgery\cite{greenlee}.


Due to recent developments in
computed tomography (CT) technology it is now possible to obtain near isotropic, submillimeter resolution images of the complete chest in a single breath hold.
This high resolution has the advantage that it enables visualisation of small
and low-contrast nodules that could hardly be screened in conventional
programs. The downside is that enormous amounts of data are generated
which increases the work load of radiologists, especially since low-dose CT
scans are more and more implemented in routine screenings. Still, this is no
idle measure. Long nodules are very commonly detected on CT scans. Research
shows that up to 51\% of smokers aged 50 years or older have pulmonary lung nodules on CT scans \cite{mahon}.
Therefore, the United States Preventive Services Task Force for example stated that it ``recommends annual screening for lung cancer with low-dose computed tomography (LDCT) in adults aged 55 to 80 years who have a 30 pack-year smoking history and currently smoke or have quit
within the past 15 years'' \cite{ups}.
Therefore, the detection of pulmonary nodules from volumetric computed
tomography (CT) scans if one of the most studied CAD applications
\cite{sluimer}.


Currently, expert radiologists perform the investigation of the
CT scans. They use the shape, the texture, the location and the growth rate of
the volume of the nodule as clinical parameter to determine the malignancy of
the nodules and to decide on the diagnosis of lung cancer. A jagged shape
nodule is more likely to be lung cancer than a smooth one. A fatty, bony,
watery nodule or a mixture of these different contents is less likely to
indicate lung cancer than a nodule that is attached to a vessel. A nodule
attached to the lung wall is typically diagnosed as benign if the
volume-doubling periode is longer than 400 days \cite{wu}. Nevertheless, the examination of these scans is a time-consuming task and is not free from errors. Although small nodules are
in principle detectable in CT scans, a non-negligible fraction may be overlooked
if they are situated in a maze of vessels of similar size \cite{ozekes}.
FIGUUR
Another problem that arises is the intra- and interreader variability
amongst radiologists in pulmonary lung detection \cite{armato} \cite{hens}. Therefore,
there is a need for a computer-aided detection (CAD) system that can assist the
radiologist in the detection of pulmonary nodules.

\subsection{The biology of lung nodules}
FIGUUR vanginneken
Lung nodules are lung tissue abnormalities that are roughly spherical with a
diameter up to 30 mm. On chest CT scans they appear as a rounded or
irregular opacity. Many types of lung nodules can by distinguished on CT scans.
A centrilobular nodule is separated by several millimeters from the pleural surfaces, fissures and
interlobular septa. They range in size from a few millimeters to 10 millimeters.
A micronodule is less than 3 millimeters in diameter. A ground-glass nodule --
or non-solid nodule -- appears on the CT scans as a hazy attenuation in the
lung. This type of nodule does not efface the bronchial and vascular margins. A solid
nodule shows a homogeneous soft-tissue attenuation. Finally, a part-solid nodule
exhibits both ground-glass and solid soft-tissue attenuation characteristics
\cite{nodule}. 

The types of nodules stated above can be categorised. Juxta-vascular
pulmonary nodules have significant connections to their neighbouring vessels.
Pleural tail nodules have only thin connections to the neighbouring pleural
wall. Well-circumscribed nodules on the other hand do not have a connection to
the neighbouring vessels and structures. Juxta-pleural nodules show some degree
of attachment to their neighbouring pleural surface \cite{kostis}.

A number of nodule segmentation algorithms perform well in detecting specific
types of nodules e.g. large, spherical, isolated nodules. However, these CAD
systems show large limitations in detecting e.g. non-isolated nodules that are
connected to the pulmonary wall \cite{keshani}. These algorithms can be usefull
in particular situations, but if a detection algorithms really aims at being an
asset for the radiologist, it should be able to detect all nodules while
refusing as much false positives as possible.

\subsection{Overview of existing lung nodule detection systems}
As the demand for a reliable CAD system to detect pulmonary nodules is urgent, a
lot of research has been dedicated to the matter. Several commercial systems
have already been developed and many workstations that radiologists use to
examinate CT scans offer on-board nodule detection or enhancement capabilities
\cite{ginneken}. However, although a lot of efforts were made, the results shown
in the various studies are rather diverse.

\subsubsection{Commercial systems}
In 2004 iCAD, Inc., provided lung cancer detection, analysis and tracking
software for the TeraRecon's Aquarius product line. The latter licensed three
software modules from iCAD. The iCAD QuickCueTM for example automatically
detects cancerous lung nodules while the iCAD QuickMatchTM locates, compares and tracks these
nodules in previous or subsequent patient studies \cite{tera}.
However, the ImageChecker CT, launched by R2 Technology, was the first CT Lung
CAD system approved by the US Food and Drug Administration for the detection of
lung nodules during the examination of CT scans \cite{Mevis}. In 2005 R2
Technology, Inc., introduced the second-generation ImageChecker CT Lung Version 2.0 CAD system which also implemented the AutoPoint temporal comparison algorithm. This CAD system ``highlights abnormalities'' and compares new and past images to
demonstrate changes that have occured over time \cite{diag} \cite{r2}.
In 2006 Vital Images, Inc., and R2 Technology, Inc., announced the
implementation of the R2 Technology's ImageChecker CT Lung CAD software into
the Vitrea workstations \cite{vital}. In 2006 anohter company, Hologic, Inc.,
acquired R2 Technology, Inc., and implemented their CAD technology
\cite{Hologic}. Then, in 2008, MeVis Medical Solutions AG, Inc., acquired the
Pulmonary Computed Tomography Business from Hologic R2, Inc. \cite{Mevis}. 


Although the technology from R2 Technology is highly demanded, some companies
have also developed their own software. In 2007 Medicsight plc announced it was
granted a medical device license from the Therapeutic Products Directorate of
Health in Canada to introduce Medicsight LungCAD API \cite{HI}.
Median Technologies offers the LMS-Lung and LMS-Lung/CAD modules which provide
quantification and detection functionalities for pulmonary (solid) nodules and
micronodules \cite{median}. And Siemens has developed the syngo.CT Lung CAD.
They claim it is ``a fully automated computer assisted second reader tool'' that is designed to assist radiologists in
the detection of solid pulmonary nodules \cite{siemens}.

\subsubsection{Automated lung detection systems: publications}
Apart from these numerous commercial systems also a lot of academic research
centra have tried to come up with a successful pulmonary nodule detection
system. \cite{review} suggest that most CAD systems for the automated detection
of lung nodules proceed according to a number of steps of which the first one is
the acquisition of data. The detection of lung nodules is preferably performed
in CT scans as they enable the visualisation of small volume and low-contrast
nodules because of the limited slice thichness. A large number of chest CT scans
are available in public databases such as the LIDC database. During the next
step, the data are pre-processed to remove noise and artefacts. This might
improve the quality of the images, but it is not necessary to do so. In the
third step a segmentation of the lungs is performed. The lung lobe region is
identified and the rest of the image is removed. This reduces the computational
cost compared to the case where the whole image is processed and it increases
the reliability, the accuracy and the precision of the algorithm. This
increases the performance of the next step: the nodule detection. ``Lung nodule
detection refers to the process of determining whether nodule patterns are
present in the image and identifying the location of the nodules''
\cite[p.~154]{review}. Nodule detection methods can be categorized according to
the detection method that is applied. The first group of publications uses a
classification technique to classify voxels or regions of interest (ROI). In
addition, a clustering method may be implemented to improve the perfromance of
the automated nodule detection method. The second group uses template matching
to detect specific geometries. The third group relies entirely on the output of
a lung nodule segmentation method. The systems that include a classification
component in their nodule detection algorithm have demonstrated better
performances \cite{review}. In the final step, the amount of false positives is
reduced to achieve a maximum sensitivity. In the following paragraphs a summary
of relevant literature is given.

\cite{elbaz} proposed an three step algorithm to isolate lung nodules from
spiral chest low-dose CT (LDCT) scans. In the first step the lung tissues were
isolated by applying an iterative Markov-Gibbs-random-field (MGRF) based
segmentation framework. To retain the nodules attached to the pulmonary walls, the
segmentation was refined by the iterative conditional mode relaxation that
maximizes a MGRF energy. Then the lung nodules, arteries, veins, bronchi and
bronchioles were separated from the rest of the tissues in the slice. In the
second step the lung nodules (2-12 mm) were detected by applying 3D and 2D
templates which describe typical geometry and greylevel distributions within
nodules of the same type. Four template shapes were used: solid sphere, hollow
sphere, solid circle and solid semicircle. The radius and the greyscale
intensity of the templates was made adaptable. The detection combined the
normalized cross-correlation template matching and a genetic optimization
algorithm. The third step eliminated the false positive nodules using three
textural and geometric features that were calculated for each detected nodule.
To distuingish between false positives and true positives Bayesian supervised
classifier learning statistical characteristics from a training set (20 FP, 20
lung TP, 20 lung wall TP nodules) of nodules selected from 50 separate subjects.
CT scans from 200 subjects were used in this study. The sensitivity was 82,3\%
and a FPNs rate of 9,2\% (i.e. the number of FPNs -- 12-- with respect to the
total number of true nodules -- 130 -- ). The speed of the execution is a
function of the CPU and the data size. The algorithm, that was implemented in
C++ on an Intel dual processor wiht 16 GB memory and 2 TB hard drive, took about
5 minutes to process 182 CT slices of size 512 x 512.



segmentation

classification
	SVM
	RF
	ANN
	
ANODE

\subsubsection{Performance of existing systems}
 The algorithms presented in a wide range of papers report varying
 successes in the automated detection of nodules. However, it is very difficult
 to compare studies against one another in a meaningful way due to differences
 in the size of the datasets, the evaluation methods, the data selection
 criteria and the characteristics of the nodules \cite{lee2010}. Especially comparing older and contemporary studies is
 difficult as older ones may have used scans with thicker sections (range 2.5 -
 10 mm), on which small nodules are rather difficult to detect, than the scans
 nowadays (2,5 mm) \cite{lee2010, ginneken} \cite{mur}. Some studies
 focus on nodules below or above a certain size or on special types of nodules
 (e.g. solid nodules). \cite{mur} performed an extended
 literature review and found that the number of scans used for testing varied between 5 and 500 with a
 median number of 29,5. Many of the studies included multiple scans from
 individual patients, which means that the diversity of the available nodules
 was reduced. Furthermore, the results of publications are often presented in a
 diverse way. \cite{results} suggests the performance of CAD systems should be
 presented a the sensitivity of the system, the specificity, the accuracy, the
 error rate, the True Positive Rate and the False Positive Rate.
 
 In order to improve the access to data, and hereby the comparability between
 studies, the Lung Image Data Consortium (LIDC) created a publically available
 database which provides researchers with a vast amount of test- and
 trainingsdata. Nevertheless, as one can take different subdatasets from this
 large database for the training and the testing of algorithms, it is still not
 possible to compare results in an objective and meaningful way. Therefore,
 \cite{ginneken} created ANODE09, a database of 55 scans and a web-based
 framework which allows researchers to test their algorithms and to
 compare results against one another.
 
\subsection{Ensemble classifiers}
- reden om classifiers te gaan gebruiken

\subsubsection{Introduction to ensemble classifiers}
voordelen/nadelen verschillende classifiers

waarom random forest

\subsubsection{Random Forests}
wisk achtergrond








%voeg hier extra sections toe

\section{Algorithms}
% TODO number of scans and slices for training and testing
The RF algorithm was trained and validated on \ldots and \ldots CT scans,
subsampled from the LIDC/IDRI database\footnote{Freely available at
\url{http://cancerimagingarchive.net}.}, consisting of \ldots and \ldots slices
respectively. The pixel size of the scans varied between 0,586 and 0,963 mm,
while the slice thickness varied between 1,25 or 2,50 mm. Together with the
original DICOM images the associated XML files were obtained. These XML files
provided a set of characteristics for each nodule found: region, subtlety,
spiculation, internal structure, lobulation, shape (sphericity), solidity,
margin, and likelihood of malignancy \cite{lidcbase}.

% TODO AANTAL NODULES IN SCANS (MAX AND MIN GROOTTE?)

The LIDC/IDRI database consists of 1018 thoracic CT scans that are obtained from
a heterogeneous range of scanner models (seven GE Medical Systems LightSpeed
scanner models, four Philips Brilleance scanner models, five Siemens Definition,
Emotion, and Sensation scanner models and one Toshiba Aquilion scanner model).
The database includes only one scan per patient so the scans are not correlated.
The nodules in the scans were delineated by at least four different expert
radiologists to indentificate as much nodules as possible. For this purpose the
indentification process was also subdivided into two phases: a blinded read
phase and an unblinded read phase. During the initial blinded read phase each
radiologist independently reviewed all scans and indicated the nodules in the
range of 3 to 30 mm, the nodules smaller than 3 mm (if not clearly benign) and
the non-nodules (other pulmonary lesions such as apical scars) larger or equal to
3 mm. In the subsequent unblinded read phase the anonymized blinded read results
of all radiologists were revealed to each of the radiologists who then
independently reviewed their marks along with the anonymous marks of their
colleagues. The delineation of the nodules was done completely manually or in a
semiautomated way. This was allowed as a study on this topic showed that the
variation in nodule delineation done by different radiologists
substantially exceeded the variation derived from different software tools
\cite{lidcbase}.
%TODO hadden we die non-nodule data moeten gebruiken?
%TODO wat is indentification?

The initial exploration of the data and the generation of a mask to perform a
lung segmentation were done in MeVisLab 2.5.1 (VC11-64) (MeVis Medical Solutions
AG, Bremen, Germany). Further processing of the data and the implementation of
the RF algorithm were carried out in Python 2.7.6 (Python Software Foundation,
Delaware, U.S.A).  The training and testing of the RF based algorithm was
performed on a computer with Intel Core i5 CPU (1,8 GHz) and 8 GB of RAM. The
processing of a medium large dataset (\ldots slices) takes \ldots minutes.
% TODO processing time

\subsection{Training of algorithm}
First the scans and the associated annotations were pre-processed in Python. The
annotations were partially provided in pixelcoordinates -- x and y values -- and
partially in worldcoordinates -- z values -- so the z coordinates were converted
into pixelcoordinates to find the nodule regions.

\subsubsection{Lung segmentation}
It was assumed that, if the whole 3D scan was fed to the cascaded classifier,
only the soft tissue would remain after the first cascade. As only one feature
-- the greyvalue of each voxel -- would be taken into account on the first level
of the cascaded classifier, this was not expected to become a problem concerning
memory storage. Unfortunately, it did pose a problem so a second option was
taken into consideration: a proper lung segmentation.

By performing a lung segmentation, the amount of voxels that have to be
processed further on is significantly reduced by about 85\%. Furthermore, it has
the advantage that the soft tissues outside the lungs are eliminated so the
accuracy of the nodule detection system is increased. Therefore, it is the first
step that is performed in a lot of papers \cite{keshani, elbaz, teramoto}. We
started with implementing a lung segmentation algorithm based on \cite{keshani}.
The first part of this algorithm consists of obtaining a binary lung CT image by
adaptive fuzzy thresholding. Then two windows of different sizes are applied to
close all the gaps in the mask and the initial lung mask is obtained by sweeping
a rotated window over the entire binary image. This sweeping is necessary to
transfer non-isolated nodules into isolated ones. Finally the mask is used to
initiate an active contour model automatically for segmenting the lung area. As
stated the first step of this algorithm was supposed to provide us with a
variable, but accurate threshold to make the binary image. The performance of
this first step was assessed by applying the algorithm on 42 slices -- 28 slices
with lungs and 14 without lungs -- equally distributed over 7 scans. The results
varied among the scans. In some cases the algorithm selected the appropriate
threshold, in other cases the soft tissue around the lungs was not eliminated
well. Instead, a fixed threshold of 1600 was emperically established to perform a
body segmentation. The lungs were not present in this body mask. Therefore, the
gaps (lungs) in the body maks were closed again by hole filling so a mask of the
entire body was obtained. As this body segmentation already eliminated 55\% of
all voxels and no complex calculations had to be done to obtain the binary image, this
result was found satisfying. Despite this reduced amount of voxels, applying
the algorithm still raised memory errors depending on the dimensions of the scan.
% TODO tabel met experimenten om vaste threshold te bepalen er nog inzetten?
% TODO duidelijker maken vanaf welk punt we afgeweken hebben van het paper?

 In order to reduce the amount of voxels even further in the pre-processing
 phase, a full lung segmentation was performed in a semiautomatic way in
 MeVisLab. After the scans were loaded in MeVisLab, the user manually indicates
 three points inside the lung area. Based on these points region growing is
 performed and a binary mask for the lung area is generated. The gaps in the
 binary mask -- which represent nodules in the lung area and nodules hidden in
 the lung wall -- are closed by dilation. This mask is then exported to Python
 for further processing of the images. An alternative way of performing a lung
 segmentation in Python would be calculating the body mask at the fixed
 threshold of 1600 and substract this mask from a similar mask but with the lung
 gaps closed. In this way only the lung area is retained. Then a dilation and
 erosion should be performed as well to include the nodules hidden in the lung
 wall.


\subsubsection{Preparation of training dataset}
Based on the associated annotations the center of gravity and the maximum radius
of each nodule was calculated. Using this information a sphere was constructed
which comprised the whole nodule. To select the central volume of the nodule,
one third of the radius was taken as an artificial boundary. Only the voxels in
this center were considered further in the process as voxels belonging to a
nodule. Reducing the amount of positive voxels -- voxels comprised by a nodule
-- was done to avoid taking into account the ambiguous edges of the nodules.
These edges might confuse the classifier. As the aim of this project was not to
delineate entire nodules but assigning nodule probabilities to the voxels in
the image, this reduction in order to provide clear training data for the
classifier was justified.

However, instead of a sphere -- which defines the nodules in 3D -- this concept
was applied per slice as it was noticed that the delineation of the nodules was
not always done properly so a lot of nodules showed a flattened shape
(\autoref{fig:flatNodule}).
% TODO maximum of minimum radius
Therefore, the maximum radii of the nodule in each slice were separately
determined and one third of these radii was taken to select the central volume
of the nodule. A list of positive voxels per scan was constructed this way.
\begin{figure}[htp]
 \begin{center}
    \includegraphics[width=90mm]{img/spherenodule_001.png}
    \caption{Flattened shape of nodule (LIDC scan 007)}
    \label{fig:flatNodule}
 \end{center}
\end{figure} %TODO use more LaTeX-friendly image format (pdf, eps)

To train a classifier, one does not only need positive example, also negative
examples are necessary. Therefore, a second list of voxels was constructed. The
amount of voxels was taken the same as in the list of positive voxels to obtain
a balanced traning dataset. The positions of the negative voxels were selected
at random over the entire image. The only constraint was that they were not
allowed the be situated within two times the maximum radius of each nodule.
%This constraint was imposed to avoid ambigous training data.

Then features were calculated for both the positive and the negative voxels. The
features that were used are discussed in \ref{sec:featureExtraction}. This
resulted in a list of features and a class (nodule or non-nodule) per voxel.
This whole process was repeated for all scans and the results were then
concatenated to obtain a dataset to train the ensemble classifier.

\subsubsection{Training Cascaded classifier}
The trainingsdataset is fed to the first level of a cascaded RF classifier. This
means that the classifier has different levels on which different amounts of
features are implemented. On a lower level, less features and less complex
features are implemented. The first feature for example is the greyvalue of each
voxel. This is a feature we get ``for free'' as it is readily available without
calculations. The algorithm simply decides on which threshold it uses to
separate nodules from non-nodules. Using this trivial feature a large part of
the voxels can already be eliminated. The voxels with a nodule probability that
is lower than a certain threshold are considered not to be nodules. These voxels
are also removed from the trainingset and only the remaining voxels are taken to
the next level in the cascaded classifier. The threshold is set very low to make
sure we do not discard positive voxels. 

The second feature is more complex and therefore it requires more computational
power. However, this is not a problem anymore as the amount of voxels to be
processed is reduced. At the second level again a number of voxels are
eliminated and removed from the dataset. The number of levels, and therefore the
number of features, can be increased until the end results are satisfying.
This however is to be decided by the user. The classifier is organised this way
as both memory and CPU time are limited. The result of this step is the creation
of a RF classifier model. This model is then tested in the next step on new data
to assess the performance.

% TODO crossvalidatie

\subsubsection{Testing Cascaded classifier}
The model that was generated in the previous step is now applied on new data to
estimate the value of the added feature in the next level of the classifier.
The question we ask ourselves here is whether a substantial amount of
non-nodule voxels are again removed from the dataset without removing the nodule
voxels as well. If this is the case the feature is kept at that level, otherwise
another feature is implemented.

The assessment is performed based on a probability image that is generated after
the model is applied on the new dataset. For each voxel in this dataset a
feature vector is calculated and this feature vector is then pushed down the
trees of the RF classifier. After each voxel is given a certain probability, the
probability image is generated.


\subsubsection{Feature extraction} \label{sec:featureExtraction}
The first feature implemented in the cascaded classifier is the greyvalue of the
voxels. The mask that was generated in MeVisLab is now projected on the image
and the greyvalues in the mask are taken into account by the classifier.

The second feature is a kind of a blobdetector: Laplacians with different sigma
values are applied on the remaining voxels. % getblurrededges?






welke features
hoe berekend

\subsection{Validation of algorithm}
nieuwe datasets van lidc ingeladen

sensitivity
specificity
FP


\subsection{Comparison with SVM}

VERGELIJKEN MET SVM?






klassendiagram: hoe is programma opgebouwd?
TRAINING + TEST
\begin{enumerate}
\item stappen die doorlopen worden + fig
\item welke algorithmes gebruikt + wat doen ze + waarom die keuze?
\item vergelijken keuzes met literatuur/commerciele systemen?
\end{enumerate}


trainingstage + teststage
tussenresultaten: moeilijkheden, opl, \ldots




\section{TODO}
\begin{algorithm}[H]
	\DontPrintSemicolon
	\caption{Training Phase\label{alg:train}}
	\ForEach{$dataset \in folder$}{
		load DICOM files\;
		load XML annotations\;
		\ForEach{$nodule \in annotations$}{
			\ForEach{$slice \in nodule$}{
				select positive voxels ($d < 0.66R_{min}$)
			}
		}
		\While {negative pixels $<$ positive pixels}
		{	select random pixel in volume\;
			\If{not too close to any nodule}{
				select negative voxel
			}
		}
	}
	\ForEach{selected pixel}{
		\For{level from 1 to max level}{
			generate feature vector up to level\;
			train classifier\;
			save classifier model\;
		}
	}
\end{algorithm}

\begin{algorithm}[H]
	\DontPrintSemicolon
	\caption{Testing \& Validation Phase\label{alg:test}}
	\ForEach{$dataset \in folder$}{
		load DICOM files\;
		mask $\longleftarrow$ load lung mask\;
		\For{level from 1 to max level}{
			load classifier model\;
			\ForEach{$pixel \in mask$}{
				generate feature vector\;
				probability $\longleftarrow$ classify\;
			}
			combine into probability image\;
			mask $\longleftarrow$ (probability image $>$ cascade threshold)\;
		}
		discard non-nodule voxels ($p < 50$\%)\;
		cluster remaining voxels\;
		load XML annotations\;
		\ForEach{$nodule \in annotations$}{
			\eIf{any cluster $\in$ nodule ($d < 2R_{min}$)}{
				$TP++$\tcc*[r]{nodule detected}
				delete cluster\;
			}{
				$FN++$\tcc*[r]{nodule not detected}
				delete cluster\;
			}
		}
		\ForEach{remaining cluster}{
			$FP++$\tcc*[r]{spurious cluster}
		}
		calculate statistics
	}
\end{algorithm}

\subsection{Statistical Measurements}
In order to evaluate the performance of a binary classifier, we introduce some
statistical concepts. The reader should be familiar with Type I and Type II
errors. A Type I error occurs when the model predicts something to be there
while in reality it is not. In this text we call these occurences false
positives (FP). In our scenario, this corresponds with a classifier indicating
that a nodule is present when there is really none.

Vice versa, a Type II error occurs when the model predicts somethig to be absent
when in reality it present. We call them false negatives (FN). False negatives
in our scenario represent nodules not detected by the classifier.

Of course the classifier does not always have to be wrong. True positives (TP)
and true negatives (TN) respresent the cases where the classifier properly
detected the presence or absence of the nodule respectively.


\autoref{tbl:stats} summarizes these definitions.
\begin{table}[htp]
\begin{center}
	\begin{tabular}{r | c c}
						& Nodule 	& Non-Nodule \\
		    \hline
		    Positive 	& TP 		& FP\\
		    Negative 	& FN 		& TN \\
	\end{tabular}
	\caption{Summary of some basic statistical measures.}
	\label{tbl:stats}
\end{center}
\end{table}

Because the terms above are in absolute numbers, they are difficult to compare
across studies. That is where sensitivity and specifictiy come in. Sensitivity
compares the amount of true positives with the total amount of positives.
Synonyms include the true positive rate or the recall rate. Specificity does the
same for the negatives. It's sometimes also called true negative rate.

\begin{equation}
	sensitivity = \frac{TP}{TP + FP}
\end{equation}

\begin{equation}
	specificity = \frac{TN}{TN + FN}
\end{equation}

Ideally both measures should be 100\%, but that is an unrealistic expectation.
There is also an inherent trade-off between the two: when the sensitivity is
increased to make sure no false positives are detected, this will also increase
the false negatives, which in turn lowers specificity. In our case there is a
clear preference for a higher sensitivity, even though it may cost us some
specificity.

One last important measure is the accuracy. It is the ratio of all correctly
classified occurrences over all occurrences.

\begin{equation}
	accuracy = \frac{TP + TN}{TP + FP + TN + FN}
\end{equation}

%TODO voxels vs clusters vs nodules

\subsection{Laplacian}
One common feature in nodule detection is the laplacian -- also called blob
detector. The laplacian operator applied to a continuous 3D function is
defined as:

\begin{equation}
	\nabla^2f(x,y,z) = \left(\frac{\partial^2 f}{\partial x^2} + \frac{\partial^2
	f}{\partial y^2} + \frac{\partial^2 f}{\partial z^2}\right)
\end{equation}

To be applied to images, it must first be discretized into a 3D convolution
mask. That mask typically has a large negative number in the center, surrounded
by positive ones.

However, because we are dealing with second derivatives, this operation is very
sensitive to noise. One solution to this problem is to convolve the image with a
gaussian kernel first. Because this kernel has a low-pass effect, noise will be
reduced.

\begin{equation}
	g(x,y,z;t) = \frac{1}{(\sqrt{2\pi} t)^3}e^{-\frac{x^2+y^2+z^2}{2t^2}}
\end{equation}

It can be represented by bionomial filters -- repeated convolutions for [1 1]
with itself -- in the discrete domain, making it rather cheap operation
compution-wise.

Convolution has some interesting properties, which allow this calculation to be
further optimized:

\begin{equation}
	\nabla^2(g * f) = (\nabla^2 g) * f %TODO continue
\end{equation}

Alternatively to the Laplacian, one can also use the Hessian matrix of second
partial derivatives as a feature. The laplacian is simply the sum of the
elements on the main diagonal. In that sense it is more complete, but also much
more computationally expensive. For this reason, we stick with the laplacian
operator.


\subsection{Cross-validation}
During the training phase, we already want to get an idea about the future
performance of our classifier. Of course we could use the whole feature set in
the training, and use the same features afterwards to check if our classifier is
performing well. However, this is considered a form of cheating, and the test
would not tell us much about the predictive power of the classifier on new data.
That is why it is important to reserve a fraction of the feature vectors for
cross-validation. A strategy called stratified K-fold is used to repeatedly
split the feature set randomly in a train and a test fraction.
The stratified adjective means that the proportion of nodules to non-nodules are
similar in both fractions. Each time -- also called \textit{fold} -- the
classifier is trained with the train fraction and performance is checked with
the test fraction. After a number of folds, these results are combined to give a
proper estimate of the classifier performance in terms of accuracy (or other
score metric).

\section{Results and discussion}
\subsection{Optimalisation of the classifier parameters}
By performing a five fould crossvalidation during the trainingstage the most
optimal parameterset for the RF algorithm was searched for.  It was found that
for all levels except the first, a value of 5 yields optimal results for the
parameter minimum samples per leaf. For level 1, this optimal value was
significantly higher at 55. %TODO why?

For each level of the classifier a threshold was set to determine which voxels
would be allowed to the next level (i.e. which voxels showed a high nodule
probability). In order to avoid discarding nodulevoxels of rather small nodules
a rather low threshold had to be set. It was empirically set at 0,1 for all
levels except for the last one. At the final level a treshold of 0,9 was set to
determine which voxels would be taken into account into our final list of
detected nodules.
 
\subsection{Validation results}
The obtained sensitivity of the algorithm was 100\% with an average of 2,17 TP
per scan and an average of 1787 FP per scan. %accuracy
% TODO mss duidelijk als je scans bekijkt welke FP zijn? TODO FPs of FP?
The algorithm shows to perform well concerning the detection of the nodules, but
by setting the thresholds at a higher value the amount of FP will probably be
reduced. Due to a lack of time the optimal thresholds were not determined
anymore. Another improvement that could be made is training the algorithm on
more datasets. To determine the optimal amount of datasets a comparison should
be made between the time it takes to train the algorithm on a certain amount of
scans and the improvements that are obtained in the performances of the
algorithm by increasing the amount of scans.

Although comparing the performances of different studies in a meaningful way is
rather difficult due to the reasons mentioned before in \ref{sec:performance}.
\cite{teramoto} showed a sensitivity of 80\% and 4,2 FP per LIDC/IDRI scan.
The detection speed was 25-34 seconds per scan. This study performed a FP reduction
by using a SVM classifier. \cite{elbaz} showed a sensitivity of 82,3\% and a
specificity of 9,2\%. The time to process 1 scan was about 5 minutes.
\cite{ginneken} compared the performances of six nodule detection CAD algorithms
on the same validation dataset. The sensitivities at seven levels of false
positive detection were calculated and then averaged. The best performing method
in this study yielded an average sensitivity of 63,2\% for the detection of all
kinds of nodules. The sensitivity per nodule type was also provided: small
nodules (63,4\%), large nodules (62,8\%), isolated nodules (60,9\%), vascular
nodules (69,3\%), pleural nodules (43,5\%) and peri-fissural nodules (76,6\%).
This clearly shows that the ease of nodule detection also depends on the type of
nodule. As this information is not available in the annotations of the scans and
as we did not cooperate with a radiologist, it is not possible to differentiate
between the different types of nodules in this project. However, as we may
assume that different nodule types are represented in our testset, it is clear
the algorithm is able to detect several types of nodules except for extremely
small ones as we removed these from the annotations in the training and
validation phase.


\begin{figure}[ht]
\begin{center}
	\begin{subfigure}[b]{\linewidth}
		\includegraphics[width=\linewidth]{img/cascades/D50S50.png}
		\caption{Slice 50}
	\end{subfigure}
	\begin{subfigure}[b]{\linewidth}
		\includegraphics[width=\linewidth]{img/cascades/D50S95.png}
  		\caption{Slice 95}
	\end{subfigure}
	\caption{Example of two slices in dataset 50, one without and one with a
	nodule.}
	\label{fig:d50}
\end{center}
\end{figure}

\begin{figure*}[p] %TODO add percentage start/left, bigger threshold at end
\begin{center}
	\begin{subfigure}[b]{0.5\linewidth}
		\begin{subfigure}[b]{\linewidth}
			\includegraphics[width=\linewidth]{img/cascades/D50L1S50.png}
			\caption{Level 1 -- Threshold: xx\% -- xx\% remaining}
		\end{subfigure}
		\begin{subfigure}[b]{\linewidth}
			\includegraphics[width=\linewidth]{img/cascades/D50L2S50.png}
			\caption{Level 2 -- Threshold: xx\% -- xx\% remaining}
		\end{subfigure}
		\begin{subfigure}[b]{\linewidth}
			\includegraphics[width=\linewidth]{img/cascades/D50L3S50.png}
			\caption{Level 3 -- Threshold: xx\% -- xx\% remaining}
		\end{subfigure}
		\begin{subfigure}[b]{\linewidth}
			\includegraphics[width=\linewidth]{img/cascades/D50L4S50.png}
			\caption{Level 4 -- Threshold: xx\% -- xx\% remaining}
		\end{subfigure}
	  \caption{Processed versions of slice 50 in dataset 50. Left: probability
	  image. Right: threshold of probability image showing the voxels that continue
	  to the next level in the cascade.}
	  \label{fig:d50s50}
  \end{subfigure}
  \begin{subfigure}[b]{0.5\linewidth}
  		\begin{subfigure}[b]{\linewidth}
			\includegraphics[width=\linewidth]{img/cascades/D50L1S95.png}
			\caption{Level 1}
		\end{subfigure}
		\begin{subfigure}[b]{\linewidth}
			\includegraphics[width=\linewidth]{img/cascades/D50L2S95.png}
			\caption{Level 2}
		\end{subfigure}
		\begin{subfigure}[b]{\linewidth}
			\includegraphics[width=\linewidth]{img/cascades/D50L3S95.png}
			\caption{Level 3}
		\end{subfigure}
		\begin{subfigure}[b]{\linewidth}
			\includegraphics[width=\linewidth]{img/cascades/D50L4S95.png}
			\caption{Level 4}
		\end{subfigure}
	  \caption{Processed versions of slice 95 in dataset 50. Left: probability
	  image. Right: threshold of probability image showing the voxels that continue
	  to the next level in the cascade. The same thresholds and remaining counts
	  apply as in figure \ref{fig:d50s50}.}
	  \label{fig:d50s95}
  \end{subfigure}
\end{center}
\end{figure*}

% \begin{figure}[p]
% \begin{center}
% 	\begin{subfigure}[b]{\linewidth}
% 		\includegraphics[width=\linewidth]{img/cascades/D50L1S95.png}
% 		\caption{Level 1}
% 	\end{subfigure}
% 	\begin{subfigure}[b]{\linewidth}
% 		\includegraphics[width=\linewidth]{img/cascades/D50L2S95.png}
% 		\caption{Level 2}
% 	\end{subfigure}
% 	\begin{subfigure}[b]{\linewidth}
% 		\includegraphics[width=\linewidth]{img/cascades/D50L3S95.png}
% 		\caption{Level 3}
% 	\end{subfigure}
% 	\begin{subfigure}[b]{\linewidth}
% 		\includegraphics[width=\linewidth]{img/cascades/D50L4S95.png}
% 		\caption{Level 4}
% 	\end{subfigure}
%   \caption{Processed versions of slice 95 in dataset 50. Left: probability
%   image. Right: threshold of probability image showing the voxels that continue
%   to the next level in the cascade. The same thresholds and remaining counts
%   apply as in figure \ref{fig:d50s50}.}
%   \label{fig:d50s95}
% \end{center}
% \end{figure}

\subsection{Suggestions for improvements}
% TODO However, the annotations assign a probability of malignancy for each
% nodule.
Separating the detected nodules into a malignancy or benignancy class is not the
main aim of this project, but this might be implemented as an extra feature.
In the ideal case, the algorithm would be able to do the processing in a couple
of minutes. This would be very interesting for a commercial software product.
However, considering the available computational power (a laptop) and the
scripting language that is used, this would not be feasible. Python is an
interpreted language which makes it inherently slower than compiled languages
such as C++. Nevertheless, Python was chosen for its rapid prototyping
abilities. Future work may implement our algorithm in C++ or another compiled
language to speed up the computational process.






\section{Conclusion}
written at end of project

\begin{enumerate}
\item what was goal project
\item what were results
\item suggestions for future research
\item \ldots
\end{enumerate}

kort stukje zelfreflectie

\clearpage
\appendix
\section{Appendix}
\subsection{Meetings}
%\includepdf[pages={-}]{meetings/1.pdf}
%\includepdf[pages={-}]{meetings/2.pdf}
%\includepdf[pages={-}]{meetings/3.pdf}
%\includepdf[pages={-}]{meetings/4.pdf}
%\includepdf[pages={-}]{meetings/5.pdf}
%\includepdf[pages={-}]{meetings/6.pdf}
%\includepdf[pages={-}]{meetings/7.pdf}
%TODO include logbook
%TODO include mission definition

\clearpage
\bibliographystyle{alpha} %plain,unsrt,alpha,abbrv,acm,apalike,siam,ieeetr,..
\bibliography{references}
\end{document}